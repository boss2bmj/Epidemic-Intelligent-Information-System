\begin{enumerate}
  \item HOSPCODE: Hospital code that is defined by the Bureau of Policy and Strategy (under Ministry of Public Health). 
  \item PID: Person identification number who registers in the hospital. This is used for identifying the person in other folders (the range of digits can be between 1 to 15 and generated by program).
  \item HN: Outpatient number that receive when the patient come to receive the service (the range of digits can be between 1 to 15)
  \item SEQ: The order of the service that be provided by the program. the number is unique, and sort by order of SEQ in each service (visit).
  \item DATE\_SERV: The date receiving the service. Defining format as year (in CE format), month and date  by order(YYYYMMDD). NOTE that in case have to record previous data, have to change the date to the date that receive the service.
  \item TIME\_SERV: Time hospital provide a service and defining format as  hour, minute, and second as (HHMMSS)  
  \item LOCATION: The area of the patient. There are two type of it which are (1) means area of responsibility and (2) means area of irresponsibility
  \item INTIME: The time that patient come and receive the service. There are two type of it which are (1) means in office hours and (2) means out of the office hours.
  \item INSTYPE: The right to medical care code that the patient use each visit.
%   ประเภทสิทธิการรักษา
%   รหัสสิทธิมาตรฐาน ที่ใชในการมารับบริการ
  \item INSID: The id number of the card of the right to medical care according to the type of the right to medical care.
%   หมายเลขของบัตร ตามประเภทสิทธิการรักษา
  \item MAIN: The standard code that come from Bureau of Policy and Strategy in order to describe the main service place.
  \item TYPEIN: It is a types of service that patient come to receive the service. There are four main types which are (1) means the patient come to receive the service by themselves, (2) means the patient come to receive the service according to the appointment, (3) means the patient that be transferred from one to other hospital, and (4) means the patient that be transferred from EMS.
  \item REFERINHOSP: The code of the hospital that send the patient to here. 
  \item CAUSEIN: The reason that transfer the patient to receive the service here. There are five main types which are (1) means to diagnose and treat, (2) means to diagnose, (3) means to treat and recover, (4) means to receive the service that close to their home, and (5) means to receive the service according to the patient's need.
  \item CHIEFCOMP: The important symptoms that patient has to come to use the service.
  \item SERVPLACE: The place that receive the service. There are 2 types which are (1) means in the service place and (2) means outside the service place
  \item BTEMP: The body temperature that measure when the patient come in. It should be in Celsius with two digits and one decimal place.
  \item SBP: Systolic pressure that first given. It should be three or less than three digits. Note that if cannot measure, not need to record
  \item DBP: Diastolic pressure that first given. It should be three or less than three digits. Note that if cannot measure, not need to record
  \item PR: Heart rate per minute. It should be three or less than three digits.
  \item RP: Breath rate per minute. It should be three or less than three digits.
  \item TYPEOUT: The status of patient after they are done receiving the service. There are nine main types which are (1) means receive drug and can go back home, (2) means receiving as Inpatient, (3) means the patient is sent to other hospital, (4) mean dead, (5) means dead before receiving the service at hospital, (6) means dead during the transfer process to other hospital, (7) means decline to treat. (8) means escape. (9) mean receive the service without diagnosis result.
  \item REFEROUTHOSP: The hospital code that transfer the patient to have a treatment.
  \item CAUSEOUT: The reason to transfer the patient to receive the service. There are five main types which are (1) means to diagnose and treat, (2) means to diagnose, (3) means to treat and recover, (4) means to receive the service that close to their home, and (5) means to receive the service according to the patient's need.
  \item COST: Total cost of treatment including drug, medical supplies, treatment. It should be eight digits and two decimal place. If it is no price, put 0.00.
  \item PRICE:Total price of treatment including drug, medical supplies, treatment. It should be eight digits and two decimal place. If it is no price, put 0.00.
  \item PAYPRICE: Total price that patient have to pay since it is the expenses that cannot be issued to the government. . It should be eight digits and two decimal place. If it is no price, put 0.00.
  \item ACTUALPAY: Total price that the patient have to pay. It should be eight digits and two decimal place. If it is no price, put 0.00.
  \item D\_UPDATE: The date in which this folder has been modified. The format is YYYYMMDDHHMMSS and the year format is CE.
  
  
\end{enumerate}